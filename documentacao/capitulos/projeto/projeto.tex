\chapter{Projeto}

Este capítulo tem como foco a apresentação ilustrativa do sistema por meio do Diagrama Entidade e Relacionamento(DER), uma representação estrutural e dos relacionamentos das classes pelo Diagrama de Classes, demonstrar as trocas de informação entre operações pelos Diagramas de Sequência e uma visão estática da estrutura física sobre a qual o software será implementado pelo Diagrama de Implantação.

\section{Diagrama Entidade e Relacionamento (DER)}

O Diagrama Entidade e Relacionamento (DER) apresenta de forma gráfica as tabelas do banco de dados, mostrando os tipos de dados que estarão armazenados, bem como os relacionamentos existentes entre as tabelas.

A Figura \ref{figura:DER} mostra as tabelas do banco de dados, onde um usuário pode morar em uma cidade, que possui um estado, e este um país, um país pode ter vários estados, um estado pode ter várias cidades e uma cidade vários usuários.

Um usuário pode ter vários álbuns, que pode ter várias multimídias. Bem como também pode ter várias postagens, que pode ter vários comentários e estes podem ter comentários também.

Uma postagem pode ter várias multimídias e várias multimídias podem estar em várias postagens, bem como uma postagem pode ter vários álbuns e vários álbuns podem estar em várias postagens.

\figura{DER}{15}{../banco_de_dados/redesocial.png}{Diagrama Entidade e Relacionamento (DER).}

\newpage

\section{Diagrama de Classes}

\section{Diagramas de Sequência}

\section{Diagrama de Implantação}
