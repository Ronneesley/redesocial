\documentclass[12pt,a4paper]{article}
\usepackage[utf8]{inputenc}
\usepackage{amsmath}
\usepackage{amsfonts}
\usepackage{amssymb}
\usepackage[brazil]{babel}
\usepackage{indentfirst}
\usepackage{url}
\usepackage{float}
\usepackage{color}
%%%%%%%%%%Codigos para o JAVA%%%%%%%%%%%%%%%%%%%%%%%%%%%%%%%%
\definecolor{pblue}{rgb}{0.13,0.13,1}
\definecolor{pgreen}{rgb}{0,0.5,0}
\definecolor{pred}{rgb}{0.9,0,0}
\definecolor{pgrey}{rgb}{0.46,0.45,0.48}
\usepackage{listings}
\lstset{language=PHP,
  showspaces=false,
  showtabs=false,
  breaklines=true,
  showstringspaces=false,
  breakatwhitespace=true,
  commentstyle=\color{pgreen},
  keywordstyle=\color{pblue},
  stringstyle=\color{pred},
  basicstyle=\ttfamily,
  moredelim=[il][\textcolor{pgrey}]{\$\$},
  moredelim=[is][\textcolor{pgrey}]{\%\%}{\%\%}
}
\lstdefinestyle{LaTeX}{
  language={[LaTeX]TeX},
  basicstyle=\ttfamily\small, 
  identifierstyle=\color{black}, 
  keywordstyle=\color{blue}, 
  commentstyle=\color{red}, 
  extendedchars=true, 
  showspaces=false, 
  showstringspaces=false, 
  numbers=none,
  breaklines=true,
  backgroundcolor=\color{yellow!20}, 
  breakautoindent=true, 
  captionpos=b,
  xleftmargin=0pt,
  frame=none,
  rframe={},
}
%%%%%%%%%%%%%%%%%%%%%%%%%%%%%%%%Fim codigo JAVA%%%%%%%%%%%%%%
%%%%%%%%%%%Codigo geral%%%%%%%%%%%%%%%%%%%%%%%%%%%%%%%%%%%%%%
\definecolor{mygreen}{rgb}{0,0.6,0}
\definecolor{mygray}{rgb}{0.5,0.5,0.5}
\definecolor{mymauve}{rgb}{0.58,0,0.82}
\lstset{ %
  backgroundcolor=\color{white},   % choose the background color; you must add \usepackage{color} or \usepackage{xcolor}; should come as last argument
  basicstyle=\footnotesize,        % the size of the fonts that are used for the code
  breakatwhitespace=false,         % sets if automatic breaks should only happen at whitespace
  breaklines=true,                 % sets automatic line breaking
  captionpos=b,                    % sets the caption-position to bottom
  commentstyle=\color{mygreen},    % comment style
  deletekeywords={...},            % if you want to delete keywords from the given language
  escapeinside={\%*}{*)},          % if you want to add LaTeX within your code
  extendedchars=true,              % lets you use non-ASCII characters; for 8-bits encodings only, does not work with UTF-8
  frame=single,                    % adds a frame around the code
  keepspaces=true,                 % keeps spaces in text, useful for keeping indentation of code (possibly needs columns=flexible)
  keywordstyle=\color{blue},       % keyword style
  language=Octave,                 % the language of the code
  morekeywords={*,...},            % if you want to add more keywords to the set
  numbers=left,                    % where to put the line-numbers; possible values are (none, left, right)
  numbersep=5pt,                   % how far the line-numbers are from the code
  numberstyle=\tiny\color{mygray}, % the style that is used for the line-numbers
  rulecolor=\color{black},         % if not set, the frame-color may be changed on line-breaks within not-black text (e.g. comments (green here))
  showspaces=false,                % show spaces everywhere adding particular underscores; it overrides 'showstringspaces'
  showstringspaces=false,          % underline spaces within strings only
  showtabs=false,                  % show tabs within strings adding particular underscores
  stepnumber=1,                    % the step between two line-numbers. If it's 1, each line will be numbered
  stringstyle=\color{mymauve},     % string literal style
  tabsize=2,                       % sets default tabsize to 2 spaces
  title=\lstname                   % show the filename of files included with \lstinputlisting; also try caption instead of title
}
%%%%%%%%%%%%%%%%%%%%%%%%%%%%%%%%Fim codigo geral%%%%%%%%%%%%%
\RequirePackage{graphicx}
\title{AJAX com Jquery}
\author{Andrey Ribeiro \and Gleydson Alvess\and Igor Justino\and Jeferson Rossini\and Warley Rodrigues de Andrade}
 
\usepackage[left=3cm,right=3cm,top=2cm,bottom=2cm]{geometry}
\begin{document}
\begin{titlepage}
\begin{center}
\begin{figure}[htb]
                
                \label{figura:LogoIF}
        
                \centering
                \includegraphics[width=6cm]{recursos/imagens/logo.png} 
\end{figure}
Instituto Federal Goiano - Campus Ceres\\
Bacharelado em Sistemas de Informação\\
Prof. Me. Ronneesley Moura Teles\\\vspace{0.5cm}
Andrey Ribeiro \\
Gleydson Alves\\
Igor Justino\\
Jeferson Rossini\\
Warley Rodrigues\\
\vspace{5.0cm}
\textit{\textbf{\Large{AJAX com JQuery}}}\\\vspace{0.5cm}
\vspace{9.5cm}
\end{center}
\end{titlepage}
\tableofcontents
\newpage
\begin{center}
\textbf{\Large{AJAX com JQuery}}\\\vspace{0.5cm}
\end{center}
\section{Introdução}
Durante o desenvolvimento de sites e sistemas web em geral, é comum a necessidade de ultilizar mecanismos do servidor sem que a página precise ser carregada novamente e para atender esse tipo de situação utilizaremos o conceito de Ajax, definido, em poucas palavras, como um conjunto de técnicas que utilizam JavaScript para carregar informações de forma assíncrona.
Ao realizar chamadas assíncronas, o fluxo do código não é interrompido até que a resposta seja obtida. Demandando sua requisição a resposta é obtida independente e tratada na função callback. Callback é um argumento de outra função, ou seja, apenas é chamada quando um evento é realizado ou quando parte do código receber uma resposta.
Os frameworks agilizam a vida do programador disponibilizando funções que fazem toda a parte complicada pedindo apenas informações básicas podendo adicionar com tranquilidade as animações enquanto a informação é carregada. O AJAX se utiliza de uma gigantesca gama de funções que o JQuery já oferece.[1]

\section{O que é AJAX?}
O Ajax (Asynchronous JavaScript and XML) é uma tecnologia que se utiliza do JavaScript, XML e HTML que é bastante evidênciada tornando aplicativos muito mais dinâmicos e agregando maiores valores na sua capacidade de resposta.
No Javascript o AJAX requer solicitações ao servidor retornando apenas os dados que a página precisa sem qualquer marcação ou apresentação. Para aqueles usuários que acessam a página que se utiliza desta tecnologia é que grande parte da página não será alterada apenas aquelas partes que serão atualizadas, com uma gigantesca diferença de antes que ao atualizar qualquer coisa a página inteira necessitaria ser carregada novamente.

\subsection{Configuracoes.class}
Configurações é o arquivo responsável por conectar com o banco de dados.
\lstinputlisting{recursos/Configuracoes.class.php}

\subsection{Funções.js}
Este é o arquivo onde está o Ajax.
\lstinputlisting{recursos/Funcoes.js}

\subsection{Pesquisa.php}
Arquivo onde recebe os dados que o ajax envia.
\lstinputlisting{recursos/pesquisa.php}



\subsection{Leis.class}
Leis.class é o arquivo responsável pelas leis: select, insert, update, etc.
\lstinputlisting{recursos/Leis.class.php}

\section{Referências Bibliográfica}
\noindent \textbf{[1]} \url {https://www.devmedia.com.br/ajax-com-jquery-trabalhando-com-requisicoes-assincronas/37141}
\\\vspace{0.2cm}

\noindent \textbf{[2]} \url {http://www.webmaster.pt/requisicoes-ajax-jquery-2216.html}

\end{document}
%Modelo de código para inserir figura
%\begin{figure}[h]
%\centering
%\includegraphics[width=15cm]{logo.png}
%\label{4}
%\caption{Fonte:http://...; Acesso em 06/11/2017}
%\end{figure}