\documentclass[12pt,a4paper]{article}
\usepackage[utf8]{inputenc}
\usepackage{amsmath}
\usepackage{amsfonts}
\usepackage{amssymb}
\usepackage[brazil]{babel}
\usepackage{indentfirst}
\usepackage{url}
\usepackage{float}
\usepackage{color}
%%%%%%%%%%Codigos para o JAVA%%%%%%%%%%%%%%%%%%%%%%%%%%%%%%%%
\definecolor{pblue}{rgb}{0.13,0.13,1}
\definecolor{pgreen}{rgb}{0,0.5,0}
\definecolor{pred}{rgb}{0.9,0,0}
\definecolor{pgrey}{rgb}{0.46,0.45,0.48}
\usepackage{listings}
\lstset{language=Java,
  showspaces=false,
  showtabs=false,
  breaklines=true,
  showstringspaces=false,
  breakatwhitespace=true,
  commentstyle=\color{pgreen},
  keywordstyle=\color{pblue},
  stringstyle=\color{pred},
  basicstyle=\ttfamily,
  moredelim=[il][\textcolor{pgrey}]{\$\$},
  moredelim=[is][\textcolor{pgrey}]{\%\%}{\%\%}
}
%%%%%%%%%%%%%%%%%%%%%%%%%%%%%%%%Fim codigo JAVA%%%%%%%%%%%%%%
%%%%%%%%%%%Codigo geral%%%%%%%%%%%%%%%%%%%%%%%%%%%%%%%%%%%%%%
\definecolor{mygreen}{rgb}{0,0.6,0}
\definecolor{mygray}{rgb}{0.5,0.5,0.5}
\definecolor{mymauve}{rgb}{0.58,0,0.82}
\lstset{ %
  backgroundcolor=\color{white},   % choose the background color; you must add \usepackage{color} or \usepackage{xcolor}; should come as last argument
  basicstyle=\footnotesize,        % the size of the fonts that are used for the code
  breakatwhitespace=false,         % sets if automatic breaks should only happen at whitespace
  breaklines=true,                 % sets automatic line breaking
  captionpos=b,                    % sets the caption-position to bottom
  commentstyle=\color{mygreen},    % comment style
  deletekeywords={...},            % if you want to delete keywords from the given language
  escapeinside={\%*}{*)},          % if you want to add LaTeX within your code
  extendedchars=true,              % lets you use non-ASCII characters; for 8-bits encodings only, does not work with UTF-8
  frame=single,                    % adds a frame around the code
  keepspaces=true,                 % keeps spaces in text, useful for keeping indentation of code (possibly needs columns=flexible)
  keywordstyle=\color{blue},       % keyword style
  language=Octave,                 % the language of the code
  morekeywords={*,...},            % if you want to add more keywords to the set
  numbers=left,                    % where to put the line-numbers; possible values are (none, left, right)
  numbersep=5pt,                   % how far the line-numbers are from the code
  numberstyle=\tiny\color{mygray}, % the style that is used for the line-numbers
  rulecolor=\color{black},         % if not set, the frame-color may be changed on line-breaks within not-black text (e.g. comments (green here))
  showspaces=false,                % show spaces everywhere adding particular underscores; it overrides 'showstringspaces'
  showstringspaces=false,          % underline spaces within strings only
  showtabs=false,                  % show tabs within strings adding particular underscores
  stepnumber=1,                    % the step between two line-numbers. If it's 1, each line will be numbered
  stringstyle=\color{mymauve},     % string literal style
  tabsize=2,                       % sets default tabsize to 2 spaces
  title=\lstname                   % show the filename of files included with \lstinputlisting; also try caption instead of title
}
%%%%%%%%%%%%%%%%%%%%%%%%%%%%%%%%Fim codigo geral%%%%%%%%%%%%%
\RequirePackage{graphicx}
\title{O que é AJAX? Como se programa?}
\author{Daniel Moreira Cardoso \and Davi Ildeu de Faria \and Fernando Maciel \and Igor Justino Rodrigues \and Paulo Henrique Rodrigues Araujo}
 
\usepackage[left=3cm,right=3cm,top=2cm,bottom=2cm]{geometry}
\begin{document}
\begin{titlepage}
\begin{center}
\begin{figure}[htb]
                
                \label{figura:LogoIF}
        
                \centering
                \includegraphics[width=6cm]{recursos/imagens/logo.png} 
\end{figure}
Instituto Federal Goiano - Campus Ceres\\
Bacharelado em Sistemas de Informação\\
Prof. Me. Ronneesley Moura Teles\\\vspace{0.5cm}
Daniel Moreira Cardoso \\
Davi Ildeu de Faria \\
Fernando Maciel \\
Igor Justino Rodrigues \\
Paulo Henrique Rodrigues Araujo \\
\vspace{5.0cm}
\textit{\textbf{\Large{O que é AJAX? Como se programa? }}}\\\vspace{0.5cm}
\vspace{9.5cm}
\end{center}
\end{titlepage}
\tableofcontents
\newpage
\begin{center}
\textbf{\Large{O que é AJAX? Como se programa?}}\\\vspace{0.5cm}
\end{center}
\section{Introdução}

O Ajax é um conjunto de tecnologias de desenvolvimento web que quando reunidas permite o desenvolvimento de uma aplicação melhor elaborada, intuitiva, e de melhor desempenho. Ajax Asynchronous JavaScript and XML(JavaScript assíncrono e XML) foi gerado por Jesse James Garret e o nome foi escolhido justamente para facilitar a explicação do modelo de interação diferente entre o navegador e o servidor web.

Asynchronous(assíncrono) é um tipo de comunicação que não ocorre exatamente ao mesmo tempo, não-simultânea. Ajax é o ato de carregar e renderizar uma página usando scripts rodando pelo lado do cliente carregando os dados em plano de fundo sem que haja a necessidade de atualização da página. Ajax não é uma tecnologia, mas sim um conjunto de tecnologias.

\subsection{Como o AJAX trabalha?}

Enquanto uma aplicação web convencional busca as informações no servidor e retorna ao cliente o AJAX não necessita dessa ação agilizando o processo de carregamento de dados, com o AJAX  toda a lógica e processamento é passado para o usuário, quando o usuário faz uma requisição quem busca e trás essas informações é o JavaScript de forma assíncrona evitando o famoso “reload” na tela. O tratamento dos dados, seu formato e exibição fica toda por conta do script que foi carregado inicialmente quando se acessou a página.

O processo inicial de carregamento é mais lento que de uma aplicação comum, pois muitas informações são pré-carregadas. Mas depois, somente os dados são carregados, tornando assim o site mais rápido.

\section{Aplicação}
A partir de um CRUD feito em PHP, para realizar as interações, realizamos um exemplo para conectar ao banco de dados e inserir sem a necessidade de atualização da página
\lstinputlisting[language=SQL]{recursos/codigo/banco.sql}
\lstinputlisting[language=PHP]{recursos/codigo/CRUD.php}
E pela página HTML enviamos os dados:
\lstinputlisting[language=HTML]{recursos/codigo/requisicao_banco_dados.html}
 
Já no próximo exemplo, informamos o nome e é retornado um alert com as horas, minutos e segundos atuais:
\lstinputlisting[language=HTML]{recursos/codigo/requisicao_dados.html}
Alert em PHP:
\lstinputlisting[language=PHP]{recursos/codigo/dados.PHP}



\section{Vantagens e Desvantagens}
\subsection{Vantagens}

\begin{itemize}
\item Uma das principais vantagens do uso do Ajax é o desempenho do site. O Ajax permite buscar  dados no servidor e mostrar em apenas uma parte da página assim não se fazendo necessário recarregar toda a página;

\item Outra vantagem bastante interessante sobre o Ajax é a compatibilidade. Não importa se a aplicação é em Java, PHP, Python ou outra linguagem de programação. (Sem o uso de frameworks);

\item Uma vantagem para o usuário é a interface. Normalmente o uso do Ajax proporciona uma melhor navegabilidade e facilidade de uso para o usuário.


\end{itemize}

\subsection{Desvantagens}

\begin{itemize}
\item Há uma boa compatibilidade com o Ajax e as linguagens de programação, porém quando usamos frameworks para implementar o Ajax podemos perder essa compatibilidade pois as vezes um framework pode interferir no funcionamento do outro. Como por exemplo a função \$ no Jquery e Prototype.

\end{itemize}

\section{Conclusão}

Com o desenvolvimento da pesquisa e implementação dos códigos Ajax foi possível perceber a real necessidade da utilização de tecnologias como esta em um mercado cada vez mais exigente e competitivo como é o nosso. Ainda contribuindo para em ganhos em desempenho e facilidade de uso (para cliente e desenvolvedor).



\section{Referências Bibliográficas}

\noindent \textbf{[1]} {https://imasters.com.br/artigo/10224/ajax/vantagens-e-desvantagens-do-uso-de-ajax-aspectos-praticos?trace=1519021197\&source=single.}\\\vspace{0.2cm}

\noindent \textbf{[2]} {https://developer.mozilla.org/pt-BR/docs/AJAX/Getting\_Started}\\\vspace{0.2cm}

\noindent \textbf{[3]} {http://www.devfuria.com.br/php/como-funcionam-os-metodos-get-e-post/}\\\vspace{0.2cm}


\end{document}
%Modelo de código para inserir figura
%\begin{figure}[h]
%\centering
%\includegraphics[width=15cm]{logo.png}
%\label{4}
%\caption{Fonte:http://...; Acesso em 06/11/2017}
%\end{figure}