\documentclass[12pt,a4paper]{article}
\usepackage[utf8]{inputenc}
\usepackage{amsmath}
\usepackage{amsfonts}
\usepackage{amssymb}
\usepackage[brazil]{babel}
\usepackage{indentfirst}
\usepackage{listings}
\usepackage{url}
\lstset{numbers=left, numberstyle=\tiny, stepnumber=1, numbersep=5pt}
\RequirePackage{graphicx}
\title{jQuery}
\usepackage[left=3cm,right=3cm,top=2cm,bottom=2cm]{geometry}

\usepackage{color}

\definecolor{mygreen}{rgb}{0,0.6,0}
\definecolor{mygray}{rgb}{0.5,0.5,0.5}
\definecolor{mymauve}{rgb}{0.58,0,0.82}

\lstset{ %
  backgroundcolor=\color{white},   % choose the background color; you must add \usepackage{color} or \usepackage{xcolor}; should come as last argument
  basicstyle=\footnotesize,        % the size of the fonts that are used for the code
  breakatwhitespace=false,         % sets if automatic breaks should only happen at whitespace
  breaklines=true,                 % sets automatic line breaking
  captionpos=b,                    % sets the caption-position to bottom
  commentstyle=\color{mygreen},    % comment style
  deletekeywords={...},            % if you want to delete keywords from the given language
  escapeinside={\%*}{*)},          % if you want to add LaTeX within your code
  extendedchars=true,              % lets you use non-ASCII characters; for 8-bits encodings only, does not work with UTF-8
  frame=single,	                   % adds a frame around the code
  keepspaces=true,                 % keeps spaces in text, useful for keeping indentation of code (possibly needs columns=flexible)
  keywordstyle=\color{blue},       % keyword style
  language=Octave,                 % the language of the code
  morekeywords={*,...},            % if you want to add more keywords to the set
  numbers=left,                    % where to put the line-numbers; possible values are (none, left, right)
  numbersep=5pt,                   % how far the line-numbers are from the code
  numberstyle=\tiny\color{mygray}, % the style that is used for the line-numbers
  rulecolor=\color{black},         % if not set, the frame-color may be changed on line-breaks within not-black text (e.g. comments (green here))
  showspaces=false,                % show spaces everywhere adding particular underscores; it overrides 'showstringspaces'
  showstringspaces=false,          % underline spaces within strings only
  showtabs=false,                  % show tabs within strings adding particular underscores
  stepnumber=1,                    % the step between two line-numbers. If it's 1, each line will be numbered
  stringstyle=\color{mymauve},     % string literal style
  tabsize=2,	                   % sets default tabsize to 2 spaces
  title=\lstname                   % show the filename of files included with \lstinputlisting; also try caption instead of title
}

\begin{document}
\begin{titlepage}


\begin{center}
\begin{figure}[htb]
		
		\label{figura:LogoIF}
	
		\centering
		\includegraphics[width=6cm]{logo.png} 
\end{figure}


Instituto Federal Goiano - Campus Ceres\\
Bacharelado em Sistemas de Informação\\
Prof. Me. Ronneesley Moura Teles\\\vspace{0.2cm}
Davi Ildeu de Faria \\
Marcos Antonio Arriel Rodrigues \\
Paulo Henrique Rodrigues Araujo \\
Salmi Nunes de Paula Junior \\
Thalia Santos de Santana\\
Warley Rodrigues de Andrade \\
Wesley Morais Felix \\
Willian Wallace de Matteus Silva\\\vspace{6.0cm}

\textit{\textbf{\Large{jQuery}}}\\\vspace{0.5cm}
\textit{\Large{Principais Funcionalidades}}\\\vspace{8.5cm}

Setembro\\
2017\\
\end{center}
\end{titlepage}



\tableofcontents

\newpage
\begin{center}
\textbf{\Large{O jQuery}}\\\vspace{0.5cm}
\end{center}
\section{O que é?}%GGN

O jQuery trata-se de uma biblioteca de JavaScript com uma série de diversas funções, capazes de ajudar desenvolvedores. Considerada como uma biblioteca popular e padrão para programação \textit{front-end} (refere-se a interfaces de \textit{websites}), é mantida pela Fundação jQuery. De acordo com o site oficial é “[...] rápida, pequena e rica em recursos”. Ademais, acaba trabalhando com a manipulação de eventos, AJAX e HTML, proporcionando uma maior integração, compatibilidade e funcionamento nos mais variados navegadores. No caso de páginas \textit{web}, auxilia principalmente para o uso de JavaScript (JS) com interações HTML.

A biblioteca foi criada em 2006, por John Resig, diante das frustrações decorrentes de códigos JS incompatíveis em navegadores diferentes. Para sua concepção, pensou-se em um agregado de funções com o intuito de resolver problemas mais complexos de serem resolvidos puramente com JavaScript.


\section{Como incluir}%GGN

Inicialmente, deve-se realizar o \textit{download} da versão mais recente (3.2.1) da biblioteca jQuery no site oficial (http://jquery.com).
Depois, é necessária sua referência no cabeçalho da página HTML, conforme código abaixo:\vspace{0.3cm}

\lstinputlisting{codigos/incluir.html} 

\vspace{0.3cm}
Ressalta-se que caminho do arquivo e o nome do \textit{.js} deve ser o mesmo usado ao baixar a biblioteca.

\section{Função do \$}

A função \$() permite selecionar elementos com maior facilidade, maior compatibilidade, com menos código e de maneira intuitiva.

\begin{itemize}
\item Código em JavaScript “Puro” \vspace{0.5cm}

\lstinputlisting{codigos/cabecalho_js.js}

O código acima adiciona um evento onde o id é “cabeçalho”. \vspace{0.5cm}

\item Código com jQuery \vspace{0.5cm}

\lstinputlisting{codigos/cabecalho_jquery.js}

A sintaxe é bem menor, e a biblioteca se encarrega de encontrar o modo mais compatível para adicionar o evento ao elemento de id “cabecalho”.

\end{itemize}

\section{Alteração de CSS}%GGN
\begin{itemize}

\item Insira seu texto aqui;
\end{itemize}

\section{Função HTML()}%GGN

\item A função HTML() tem dois usos distintos:
\\
\\
1 - Pode-se usá-la para obter o conteúdo em HTML de um determinado elemento;
\\
\\
2 - Para definir esse conteúdo;
\\
\\
Na primeira forma, basta que o método seja invocado sem a passagem de parâmetros, no segunda deve-se passar o conteúdo total do elemento.

\section{Função Append()}%GGN
\begin{itemize}

\item Insira seu texto aqui;
\end{itemize}


\section{Referências}
\begin{enumerate}

\item KHAN ACADEMY. \textbf{Pausa para a história: como John criou o jQuery?} Disponível em: $<$http://https://pt.khanacademy.org/computing/computer-pro
gramming/html-js-jquery/jquery-dom-access/a/history-of-jquery$>$. Acesso em: 22 de setembro de 2017.

\item JQUERY. \textbf{jQuery, write less, do more.} Disponível em: $<$http://jquery.com$>$. Acesso em: 21 de setembro de 2017.

\item CAELUM. \textbf{Apostila Desenvolvimento Web com HTML, CSS e JavaScript.} Disponível em: $<$https://www.caelum.com.br/apostila-html-css-javascript/jquery/$>$. Acesso em: 22 de setembro de 2017.

\item W3SCHOOLS. \textbf{jQuery html() Method.} Disponível em: $<$https://www.w3
schools.com/jquery/html$\_$html.asp$>$. Acesso em: 28 de setembro de 2017.

\end{enumerate}

\end{document}