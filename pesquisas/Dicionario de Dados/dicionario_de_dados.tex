\documentclass[12pt,a4paper]{article}
\usepackage[utf8]{inputenc}
\usepackage{amsmath}
\usepackage{amsfonts}
\usepackage{amssymb}
\usepackage[brazil]{babel}
\usepackage{indentfirst}
\usepackage{url}
\usepackage{float}
\usepackage{color}
\usepackage{tabela_de_dados}
\usepackage{colortbl}

\definecolor{pblue}{rgb}{0.13,0.13,1}
\definecolor{pgreen}{rgb}{0,0.5,0}
\definecolor{pred}{rgb}{0.9,0,0}
\definecolor{pgrey}{rgb}{0.46,0.45,0.48}
\definecolor{cinzaClaro}{rgb}{0.9,0.9,0.9}

\definecolor{mygreen}{rgb}{0,0.6,0}
\definecolor{mygray}{rgb}{0.5,0.5,0.5}
\definecolor{mymauve}{rgb}{0.58,0,0.82}

\RequirePackage{graphicx}
\title{Dicionário de Dados}
\author{Gusttavo Nunes Gomes\and Jonathan Silvestre Sousa \and Salmi Nunes de Paula Junior\and Willian Wallace de Matteus Silva}
 
\usepackage[left=3cm,right=3cm,top=2cm,bottom=2cm]{geometry}
\begin{document}
\begin{titlepage}
\begin{center}
\begin{figure}[htb]
                
                \label{figura:LogoIF}
        
                \centering
                \includegraphics[width=6cm]{recursos/imagens/logo.png} 
\end{figure}
Instituto Federal Goiano - Campus Ceres\\
Bacharelado em Sistemas de Informação\\
Prof. Me. Ronneesley Moura Teles\\\vspace{1cm}

Gusttavo Nunes Gomes\\
Jonathan Silvestre Sousa \\
Salmi Nunes de Paula Junior\\
Willian Wallace de Matteus Silva\\
\vspace{6.0cm}
\textit{\textbf{\Large{Dicionário de Dados}}}\\\vspace{11cm}
Novembro\\
2017\\
\end{center}
\end{titlepage}
\tableofcontents
\newpage
\begin{center}
\textbf{\Large{JDBC com transações}}\\\vspace{0.5cm}
\end{center}
\section{Introdução}

\begin{tabbing}
\hspace{1.0cm}\=\hspace{2cm}\=\hspace{0.5cm}\=\hspace{2cm}\=\hspace{2cm}\=\hspace{2cm}\=\kill
 • \>  • \>  • \>  • \>  • \>  • \> • \\ 
 • \>  • \>  • \>  • \>  • \>  • \> • \\ 
 • \>  • \>  • \>  • \>  • \>  • \> •
\end{tabbing} 

\begin{center}

\begin{tabular}{|p{2cm}|p{2cm}|p{2cm}|p{2cm}|p{2cm}|p{2cm}|p{2cm}|}
\hline 
Chaves & Campo Lógico & Campo Físico & Tipo & Tamaho & Nulo & Descrição\\
\hline 
PK & Campo Lógico & Campo Físico & Tipo & Tamaho & Nulo & Descrição\\
\hline 

\end{tabular} 
\end{center}

\begin{center}


\begin{table}[h!]
		\caption{Caso de Uso: 2}
		\label{tabela:1}
		\begin{tabular}{|p{1cm}|p{2cm}|p{2cm}|p{1cm}|p{1.5cm}|p{2cm}|p{3cm}|}
		\hline
		
				
			\multicolumn{7}{|p{16cm}|}{\cellcolor{cinzaClaro} Tabela: } \\ 
			\hline
		{\small Chaves} & {\small Campo Lógico} & {\small Campo Físico} & {\small Tipo} & {\small Tamanho} & {\small Nulo} & {\small Descrição}\\
\hline 
		
		{\small ID} & {\small ID} & {\small ID} & {\small VARCHAR} & {\small 1000} & {\small Not Null} & {\small dASDASDASFAS ASDASD ASD ASD ASDSA} \\\hline 		
		
		{\small ID} & {\small ID} & {\small ID} & {\small ID} & {\small ID} & {\small ID} & {\small ID} \\\hline 
		ID & {\small ID} & {\footnotesize ID} & {\scriptsize ID} & {\tiny ID} & ID &
			
		\end{tabular}
	\end{table}
	
	\end{center}

\newpage
\section{Referências Bibliográfica}
\noindent \textbf{[1]} \url {http:/}\\\vspace{0.2cm}

\noindent \textbf{[2] }\url{https://pt.wikipedia.org/wiki/JDBC}\\\vspace{0.2cm}
\end{document}
%Modelo de código para inserir figura
%\begin{figure}[h]
%\centering
%\includegraphics[width=15cm]{logo.png}
%\label{4}
%\caption{Fonte:http://...; Acesso em 06/11/2017}
%\end{figure}