\documentclass[12pt,a4paper]{article}
\usepackage[utf8]{inputenc}
\usepackage{amsmath}
\usepackage{amsfonts}
\usepackage{amssymb}
\usepackage[brazil]{babel}
\usepackage{indentfirst}
\usepackage{url}
\usepackage{listings}
\RequirePackage{graphicx}
\title{Algoritmo Viola-Jones}
\author{Adallberto Lucena \and Andrey Silva \and Anny Karoliny \and Brener Gomes \and Davi Ildeu \and Eduardo de Oliveira \and Gleyson Israel \and Gusttavo Nunes \and Ianka Talita \and Ígor Justino}

 
\usepackage[left=3cm,right=3cm,top=2cm,bottom=2cm]{geometry}


\begin{document}
\begin{titlepage}


\begin{center}
\begin{figure}[htb]
		
		\label{figura:LogoIF}
	
		\centering
		\includegraphics[width=6cm]{logo.png} 
\end{figure}


Instituto Federal Goiano - Campus Ceres\\
Bacharelado em Sistemas de Informação\\
Prof. Me. Ronneesley Moura Teles\\\vspace{0.5cm}
Adallberto Lucena Moura \\
Andrey Silva Ribeiro\\
Anny Karoliny Moraes Ribeiro \\
Brener Gomes de Jesus\\
Davi Ildeu de Faria\\
Eduardo de Oliveira Silva\\
Gleyson Israel Alves\\
Gusttavo Nunes Gomes\\
Ianka Talita Bastos de Assis\\
Ígor Justino Rodrigues\\



\vspace{5.0cm}

\textit{\textbf{\Large{Algoritmo Viola-Jones}}}\\\vspace{0.5cm}
\vspace{9.5cm}

Outubro\\
2017\\
\end{center}
\end{titlepage}



\tableofcontents

\newpage
\begin{center}
\textbf{\Large{Algoritmo Viola-Jones}}\\\vspace{0.5cm}
\end{center}

\section{O artigo}
Em 2001, \textit{Paul Viola} e \textit{Michael Jones}, dois pesquisadores de Cambridge publicaram uma artigo entitulado: “\textit{Rapid Object Detection using a Boosted Cascade of Simple Features}” que demonstrava um novo método de detecção de faces. O artigo se diferencia e se paltava em 3 pontos importantes. 

A primeira foi uma nova maneira de se representar uma imagem, a “imagem integral” (\textit{Integral Image}, em inglês), que permitiu os detectores usados por eles, computarem a imagem de maneira mais rápida.

A segunda foi o algoritmo de aprendizado baseado no \textit{AdaBoost}, que selecionava um número pequeno de características visuais críticas de um conjunto maior e com seus classificadores, extremamente eficientes.

O terceiro aspecto importante, foi o método de combinar e incrementar classificadores em “cascata”, o que permitia regiões do fundo da foto de serem rapidamente descatadas, disponibilizando maior processamento computacional em posições com maior possíbilidade de ser o objeto no qual se está procurando, como um rosto.

Os autores aprofundaram seus métodos de como construiram esse algoritmo, de como funcionavam as equações e apresentaram os resultados encontrados e compararam com algoritmos similares da época, e como os algoritmos “\textit{Rowley-Baluja-Kanade}”, “\textit{Schneiderman-Kanade}” e “\textit{Roth-Yang-Ahuja}”.

Devido sua implementação seguir uma abordagem diferente para  construção de um sistema de detecção de face, aproximadamente 15 vezes mais rápida que os métodos anteriores, o algoritmo \textit{Viola-Jones} revolucionou esse campo da computação se tornando uma referência.

\section{Funcionamento do algoritmo}



\section{Vantagem}
\begin{itemize}
	\item 15 vezes mais rápido que o algoritmo “\textit{Rowley-Baluja-Kanade}” no processamento da imagem.

	\item 600 vezes se comparado ao “\textit{Schneiderman-Kanade}”.
\end{itemize}

\section{Desvantagem}
\begin{itemize}
	\item A detecção de faces, só é possível se o rosto estiver na posição frontal.
	\item A base de dados usada, precisa de faces em diferentes condições incluindo: iluminação, brilho, escala, pose e variações de câmera.
	\item Nível de detecção na literatura - 80\% (FAUX,2012)
	\item É um algoritmo de detecção de face e não de reconhecimento facial.
\end{itemize}

\section{Exemplos de Implementação}

\subsection{Python}
\url{https://github.com/Simon-Hohberg/Viola-Jones}


 





\newpage
\section{Referências Bibliográfica}
\noindent VIOLA, P. e JONES, M.\textbf{ Rapid object detection using a boosted cascade of simple features.} Proceedings of the 2001 IEEE Computer Society Conference on Computer Vision and Pattern Recognition. CVPR 2001, v. 1, p. I-511-I-518, 2001. Disponível em: $<$\url{http://ieeexplore.ieee.org/document/990517/}$>$.\\\vspace{0.2cm}

\noindent IRGENS, Peter et al. \textbf{An efficient and cost effective FPGA based implementation of the Viola-Jones face detection algorithm.} HardwareX, v. 1, p. 68–75, 2017. Disponível em: $<$\url{http://linkinghub.elsevier.com/retrieve/pii/S2468067216300116}$>$.\\\vspace{0.2cm}

\noindent SANTOS, Ligneul. \textbf{Detecção de faces através do algoritmo de Viola-Jones}. Coppe/Ufrj, 2011.\\\vspace{0.2cm}

\noindent FAUX, Francis e LUTHON, Franck. \textbf{Theory of evidence for face detection and tracking}. International Journal of Approximate Reasoning, v. 53, n. 5, p. 728–746, 2012. Disponível em: $<$\url{http://dx.doi.org/10.1016/j.ijar.2012.02.002}$>$.\\\vspace{0.2cm}

\noindent BODHI, S. R. e NAVEEN, S. \textbf{Face detection, registration and feature localization experiments with RGB-D face database}. Procedia Computer Science, v. 46, n. Icict 2014, p. 1778–1785, 2015. Disponível em: $<$\url{http://dx.doi.org/10.1016/j.procs.2015.02.132}$>$.





\end{document}

%Modelo de código para inserir figura

%\begin{figure}[h]
%\centering
%\includegraphics[width=15cm]{logo.png}
%\label{4}
%\caption{Fonte:http://...; Acesso em 07/10/2017}
%\end{figure}