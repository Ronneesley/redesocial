\documentclass[12pt,a4paper]{article}
\usepackage[utf8]{inputenc}
\usepackage{amsmath}
\usepackage{amsfonts}
\usepackage{amssymb}
\usepackage[brazil]{babel}
\usepackage{indentfirst}
\usepackage{url}
\usepackage{listings}
\usepackage{color}
\lstset{language=Python}
\definecolor{mygreen}{rgb}{0,0.6,0}
\definecolor{mygray}{rgb}{0.5,0.5,0.5}
\definecolor{mymauve}{rgb}{0.58,0,0.82}

\lstset{ %
  backgroundcolor=\color{white},   % choose the background color; you must add \usepackage{color} or \usepackage{xcolor}; should come as last argument
  basicstyle=\footnotesize,        % the size of the fonts that are used for the code
  breakatwhitespace=false,         % sets if automatic breaks should only happen at whitespace
  breaklines=true,                 % sets automatic line breaking
  captionpos=b,                    % sets the caption-position to bottom
  commentstyle=\color{mygreen},    % comment style
  deletekeywords={...},            % if you want to delete keywords from the given language
  escapeinside={\%*}{*)},          % if you want to add LaTeX within your code
  extendedchars=true,              % lets you use non-ASCII characters; for 8-bits encodings only, does not work with UTF-8
  frame=single,	                   % adds a frame around the code
  keepspaces=true,                 % keeps spaces in text, useful for keeping indentation of code (possibly needs columns=flexible)
  keywordstyle=\color{blue},       % keyword style
  language=Octave,                 % the language of the code
  morekeywords={*,...},            % if you want to add more keywords to the set
  numbers=left,                    % where to put the line-numbers; possible values are (none, left, right)
  numbersep=5pt,                   % how far the line-numbers are from the code
  numberstyle=\tiny\color{mygray}, % the style that is used for the line-numbers
  rulecolor=\color{black},         % if not set, the frame-color may be changed on line-breaks within not-black text (e.g. comments (green here))
  showspaces=false,                % show spaces everywhere adding particular underscores; it overrides 'showstringspaces'
  showstringspaces=false,          % underline spaces within strings only
  showtabs=false,                  % show tabs within strings adding particular underscores
  stepnumber=1,                    % the step between two line-numbers. If it's 1, each line will be numbered
  stringstyle=\color{mymauve},     % string literal style
  tabsize=2,	                   % sets default tabsize to 2 spaces
  title=\lstname                   % show the filename of files included with \lstinputlisting; also try caption instead of title
}
\RequirePackage{graphicx}
\title{Algoritmo Viola-Jones}
\author{Adallberto Lucena \and Andrey Silva \and Anny Karoliny \and Brener Gomes \and Davi Ildeu \and Eduardo de Oliveira \and Gleyson Israel \and Gusttavo Nunes \and Ianka Talita \and Ígor Justino}

 
\usepackage[left=3cm,right=3cm,top=2cm,bottom=2cm]{geometry}


\begin{document}
\begin{titlepage}


\begin{center}
\begin{figure}[htb]
		
		\label{figura:LogoIF}
	
		\centering
		\includegraphics[width=6cm]{logo.png} 
\end{figure}


Instituto Federal Goiano - Campus Ceres\\
Bacharelado em Sistemas de Informação\\
Prof. Me. Ronneesley Moura Teles\\\vspace{0.5cm}
Adallberto Lucena Moura \\
Andrey Silva Ribeiro\\
Anny Karoliny Moraes Ribeiro \\
Brener Gomes de Jesus\\
Davi Ildeu de Faria\\
Eduardo de Oliveira Silva\\
Gleyson Israel Alves\\
Gusttavo Nunes Gomes\\
Ianka Talita Bastos de Assis\\
Ígor Justino Rodrigues\\



\vspace{5.0cm}

\textit{\textbf{\Large{Algoritmo Viola-Jones}}}\\\vspace{0.5cm}
\vspace{9.5cm}

Outubro\\
2017\\
\end{center}
\end{titlepage}



\tableofcontents

\newpage
\begin{center}
\textbf{\Large{Algoritmo Viola-Jones}}\\\vspace{0.5cm}
\end{center}

\section{O artigo}
Em 2001, \textit{Paul Viola} e \textit{Michael Jones}, dois pesquisadores de Cambridge publicaram uma artigo entitulado: “\textit{Rapid Object Detection using a Boosted Cascade of Simple Features}” que demonstrava um novo método de detecção de faces. O artigo se diferencia e se paltava em 3 pontos importantes. 

A primeira foi uma nova maneira de se representar uma imagem, a “imagem integral” (\textit{Integral Image}, em inglês), que permitiu os detectores usados por eles, computarem a imagem de maneira mais rápida.

A segunda foi o algoritmo de aprendizado baseado no \textit{AdaBoost}, que selecionava um número pequeno de características visuais críticas de um conjunto maior e com seus classificadores, extremamente eficientes.

O terceiro aspecto importante, foi o método de combinar e incrementar classificadores em “cascata”, o que permitia regiões do fundo da foto de serem rapidamente descatadas, disponibilizando maior processamento computacional em posições com maior possíbilidade de ser o objeto no qual se está procurando, como um rosto.

Os autores aprofundaram seus métodos de como construiram esse algoritmo, de como funcionavam as equações e apresentaram os resultados encontrados e compararam com algoritmos similares da época, e como os algoritmos “\textit{Rowley-Baluja-Kanade}”, “\textit{Schneiderman-Kanade}” e “\textit{Roth-Yang-Ahuja}”.

Devido sua implementação seguir uma abordagem diferente para  construção de um sistema de detecção de face, aproximadamente 15 vezes mais rápida que os métodos anteriores, o algoritmo \textit{Viola-Jones} revolucionou esse campo da computação se tornando uma referência.

\section{Funcionamento do algoritmo}



\section{Vantagem}
\begin{itemize}
	\item 15 vezes mais rápido que o algoritmo “\textit{Rowley-Baluja-Kanade}” no processamento da imagem.

	\item 600 vezes se comparado ao “\textit{Schneiderman-Kanade}”.
\end{itemize}

\section{Desvantagem}
\begin{itemize}
	\item A detecção de faces, só é possível se o rosto estiver na posição frontal.
	\item A base de dados usada, precisa de faces em diferentes condições incluindo: iluminação, brilho, escala, pose e variações de câmera.
	\item Nível de detecção na literatura - 80\% (FAUX,2012)
	\item É um algoritmo de detecção de face e não de reconhecimento facial.
\end{itemize}

\section{Implementação Viola-Jones em Python}

\subsection{Porquê a implementação em Python?}
Existem inúmeras implementações do algoritmo na internet, nas mais diversas linguagens, sendo mais comuns em \textit{MatLab}, \textit{Java} e \textit{Python}. 

A implementação em MatLab não era muito interessante devido a ferramenta em si, é uma boa ferramenta para desenvolver procedimentos tais como esse de detecção facial. entretanto é uma ferramenta paga,e  em um projeto nessa escala, no qual muitas pessoas iriam interagir, não ter a licença de uso, dificultaria muito o processo, e por isso foi descartada.

\textit{Java} era muito abundante no \textit{GitHub}, entretanto a maioria estava incompleta, não possuindo os códigos para treinar a rede, o que é fundamental para o sucesso do algoritmo, ter uma base de dados ampla gerando uma maior eficiência na detecção de imagens. Os códigos que estavam completos, a documentação era quase inexistente, o que dificultaria a utilização, sendo talvezmais vantajoso desenvolver desde o início uma amplicação do que gastar tempo em código alheio. 


Entretanto escolhemos a implementação de \textit{Simon Hohberg} do algoritmo em \textit{Python} disponibilizado no seu \textit{GitHub} pessoal:
\url{https://github.com/Simon-Hohberg/Viola-Jones}


\textit{Python} é uma linguagem mais simples, mas nem por isso menos robusta, é amplamente usada em processos de \textit{machine learning} e a implementação encontrada, está com todas as partes do algoritmo, desde a parte de detecção até a de treinamento, e a documentação do código está bem feita, o que possibilitaria o uso.

A seguir segue os principais códigos do algoritmo implementado em \textit{Python} por \textit{Simon Hohberg}:
\url{https://github.com/Simon-Hohberg/Viola-Jones}



\subsection{IntegralImage.py}

 \lstinputlisting[language=Python, firstline=1, lastline=58]{recursos/codigo_python/Viola-Jones-master/violajones/IntegralImage.py}
 
 \subsection{HaarLikeFeature.py}

 \lstinputlisting[language=Python, firstline=1, lastline=89]{recursos/codigo_python/Viola-Jones-master/violajones/HaarLikeFeature.py}

\subsection{AdaBoost.py}
\lstinputlisting[language=Python, firstline=1, lastline=122]{recursos/codigo_python/Viola-Jones-master/violajones/AdaBoost.py}

\subsection{Utils.py}

\lstinputlisting[language=Python, firstline=1, lastline=108]{recursos/codigo_python/Viola-Jones-master/violajones/Utils.py}

\subsection{example.py}

\lstinputlisting[language=Python, firstline=1, lastline=108]{recursos/codigo_python/Viola-Jones-master/example.py}


%\lstinputlisting[language=Python, firstline=37, lastline=45]{source_filename.py}


\newpage
\section{Referências Bibliográfica}
\noindent VIOLA, P. e JONES, M. \textbf{Rapid object detection using a boosted cascade of simple features.} Proceedings of the 2001 IEEE Computer Society Conference on Computer Vision and Pattern Recognition. CVPR 2001, v. 1, p. I-511-I-518, 2001. Disponível em: $<$\url{http://ieeexplore.ieee.org/document/990517/}$>$.\\\vspace{0.2cm}

\noindent VIOLA, P. e JONES, M. \textbf{Robust Real-Time Face
Detection}  International Journal of Computer Vision
57(2), 137–154, 2004.\\\vspace{0.2cm}


\noindent IRGENS, Peter et al. \textbf{An efficient and cost effective FPGA based implementation of the Viola-Jones face detection algorithm.} HardwareX, v. 1, p. 68–75, 2017. Disponível em: $<$\url{http://linkinghub.elsevier.com/retrieve/pii/S2468067216300116}$>$.\\\vspace{0.2cm}

\noindent SANTOS, Ligneul. \textbf{Detecção de faces através do algoritmo de Viola-Jones}. Coppe/Ufrj, 2011.\\\vspace{0.2cm}

\noindent FAUX, Francis e LUTHON, Franck. \textbf{Theory of evidence for face detection and tracking}. International Journal of Approximate Reasoning, v. 53, n. 5, p. 728–746, 2012. Disponível em: $<$\url{http://dx.doi.org/10.1016/j.ijar.2012.02.002}$>$.\\\vspace{0.2cm}

\noindent BODHI, S. R. e NAVEEN, S. \textbf{Face detection, registration and feature localization experiments with RGB-D face database}. Procedia Computer Science, v. 46, n. Icict 2014, p. 1778–1785, 2015. Disponível em: $<$\url{http://dx.doi.org/10.1016/j.procs.2015.02.132}$>$.







\end{document}

%Modelo de código para inserir figura

%\begin{figure}[h]
%\centering
%\includegraphics[width=15cm]{logo.png}
%\label{4}
%\caption{Fonte:http://...; Acesso em 07/10/2017}
%\end{figure}