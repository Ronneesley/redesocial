\documentclass[14pt, a4paper]{article}
\usepackage{xcolor}
% Definindo novas cores
\definecolor{verde}{rgb}{0.25,0.5,0.35}
\definecolor{jpurple}{rgb}{0.5,0,0.35}
% Configurando layout para mostrar codigos Java
\usepackage{listings}
\lstset{
  language=Java,
  basicstyle=\ttfamily\small, 
  keywordstyle=\color{jpurple}\bfseries,
  stringstyle=\color{red},
  commentstyle=\color{verde},
  morecomment=[s][\color{blue}]{/**}{*/},
  extendedchars=true, 
  showspaces=false, 
  showstringspaces=false, 
  numbers=left,
  numberstyle=\tiny,
  breaklines=true, 
  backgroundcolor=\color{cyan!10}, 
  breakautoindent=true, 
  captionpos=b,
  xleftmargin=0pt,
  tabsize=4
}
\pagestyle{empty}
\usepackage[utf8]{inputenc}
\usepackage{amsmath}
\usepackage{amsfonts}
\usepackage{amssymb}
\usepackage[brazil]{babel}
\usepackage{indentfirst}
\usepackage{url}

\RequirePackage{graphicx}
\title{Class Runtime}
\author{Eduardo Oliveira Silva \and Jônatas de Souza Rezende \and Lucas Pereira de Azevedo \and Maciele Xavier Rodrigues}
\usepackage[left=3cm,right=3cm,top=2cm,bottom=2cm]{geometry}

\begin{document}

\begin{titlepage}


\begin{center}
\begin{figure}[htb]
		
		\label{figura:LogoIF}
	
		\centering
		\includegraphics[width=6cm]{logo.png} 
\end{figure}


Instituto Federal Goiano - Campus Ceres\\
Bacharelado em Sistemas de Informação\\
Prof. Me. Ronneesley Moura Teles\\\vspace{0.2cm}
Eduardo Oliveira Silva \\
Jônatas de Souza Rezende \\
Maciele Savier Rodrigues \\
Lucas Pereira de Azevedo \\\vspace{7.0cm}

\textit{\textbf{\Large{Recuperação de Senha}}}\\\vspace{0.5cm}
\textit{\textbf{\Large{Enviar e-mail pelo Java}}}\\\vspace{9.5cm}

Outubro\\
2017\\
\end{center}
\end{titlepage}

\tableofcontents

\newpage
\begin{center}
\textbf{\Large{Recuperação de Senha - Enviar e-mail pelo Java}}\\\vspace{0.5cm}
\end{center}
\section{Procecimento para recuperação}%GGN

Para recuperar a senha geralmente é usado o envio de um e-mail solicitando ao usuário a criação de uma nova senha, nesse e-mail é apresentado um link onde se realmente o usuário solicitou a recuperação da senha deve clicar para que possa ser providenciado a criação e envio de um novo e-mail com a senha provisória.

A forma de enviar e-mail usando o Java se encontra explicada nos capítulos seguintes.

\section{Código de exemplo}%GGN

\begin{lstlisting}
import java.util.Properties;
import javax.mail.Address;
import javax.mail.Message;
import javax.mail.MessagingException;
import javax.mail.PasswordAuthentication;
import javax.mail.Session;
import javax.mail.Transport;
import javax.mail.internet.InternetAddress;
import javax.mail.internet.MimeMessage;

public class JavaMailApp
{
      public static void main(String[] args) {
            Properties props = new Properties();
            props.put("mail.smtp.host", "smtp.gmail.com");
            props.put("mail.smtp.socketFactory.port", "465");
            props.put("mail.smtp.socketFactory.class", "javax.net.ssl.SSLSocketFactory");
            props.put("mail.smtp.auth", "true");
            props.put("mail.smtp.port", "465");

            Session session = Session.getDefaultInstance(props,
                        new javax.mail.Authenticator() {
                             protected PasswordAuthentication getPasswordAuthentication() 
                             {
                                   return new PasswordAuthentication("testarjavamail@gmail.com", "tjm123456");
                             }
                        });
            session.setDebug(true);
            try {

                  Message message = new MimeMessage(session);
                  message.setFrom(new InternetAddress("testarjavamail@gmail.com"));

                  Address[] toUser = InternetAddress
                             .parse("testarjavamail@gmail.com");  
                  message.setRecipients(Message.RecipientType.TO, toUser);
                  message.setSubject("Enviando email com JavaMail");
                  message.setText("Enviei este email utilizando JavaMail com minha conta GMail!");
                  Transport.send(message);
                  System.out.println("Feito!!!");
             } catch (MessagingException e) {
                  throw new RuntimeException(e);
            }
      }
}
\end{lstlisting}

\section{Explicação do código.}

\begin{enumerate}
\item Importações necessárias para a execução do código Java:

\begin{lstlisting}
import java.util.Properties;
import javax.mail.Address;
import javax.mail.Message;
import javax.mail.MessagingException;
import javax.mail.PasswordAuthentication;
import javax.mail.Session;
import javax.mail.Transport;
import javax.mail.internet.InternetAddress;
import javax.mail.internet.MimeMessage;
\end{lstlisting}

\item O objeto \textbf{props} define os parâmetros da conexão com o servidor Gmail.

\begin{lstlisting}
Properties props = new Properties();
props.put("mail.smtp.host", "smtp.gmail.com");
props.put("mail.smtp.socketFactory.port", "465");
props.put("mail.smtp.socketFactory.class","javax.net.ssl.SSLSocketFactory");
props.put("mail.smtp.auth", "true");
props.put("mail.smtp.port", "465");
\end{lstlisting}

\item O objeto \textbf{session} inicia a sessão para autenticação do usuário remetente do e-mail
        
\begin{lstlisting}
Session session = Session.getDefaultInstance(props, new javax.mail.Authenticator() {
	protected PasswordAuthentication getPasswordAuthentication()
	{
	return new PasswordAuthentication("testarjavamail@gmail.com", "tjm123456");
    }
});

\end{lstlisting}

\item Para facilitar o acompanhamento da execução do código é ativado o debug para a sessão.

\begin{lstlisting}          
session.setDebug(true);
\end{lstlisting}                

\item O objeto \textbf{message} define as informações do e-mail.

\begin{lstlisting}
Message message = new MimeMessage(session);
message.setFrom(new InternetAddress("testarjavamail@gmail.com"));
Address[] toUser = InternetAddress.parse("testarjavamail@gmail.com");  
message.setRecipients(Message.RecipientType.TO, toUser);
message.setSubject("Enviando email com JavaMail");
message.setText("Enviei este email utilizando JavaMail com minha conta GMail!");

\end{lstlisting}                    

\item Método para enviar a mensagem criada.

\begin{lstlisting}
Transport.send(message);
\end{lstlisting}                                        

\end{enumerate}

\section{Referências}
\begin{enumerate}

\item ORACLE. \textbf{JavaMail.} Disponível em: $<$http://www.oracle.com/technetwork/java/javamail/index.html$>$. Acesso em: 23 de outubro de 2017.

\end{enumerate}

\end{document}


\end{document}